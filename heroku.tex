%\usepackage{hyperref}  ESTO ME FALLA Y NO SE PORQUÉ


\section{Empezar con Heroku} %{\tt system}}
\subsection{Crear cuenta en Heroku} 
Lo primero que tenemos que hacer es crearnos una cuenta en Heroku. Para ello nos dirigimos a la web de \href{www.heroku.com}{Heroku} y pulsamos 
en \textbf{Login} y pulsar en  \textbf{Sign up}.Nos pedirá que introduzcamos una cuenta de correo y pulsemos en \textbf{Sign up}. 
Ahora tendremos que revisar nuestra bandeja de entrada y pulsar el link que aparece en el correo recibido para poder confirmar nuestra cuenta.
Una vez confirmada, tendremos nuestra cuenta creada.

\subsection{Instalar Heroku Toolbelt}

El siguiente paso es instalar Heroku Toolbelt en nuestro Sistema Operativo. Heroku Toolbelt está compuesto de un cliente de Heroku y Foreman 
(opción que nos facilita correr nuestras aplicaciones localmente). En nuestro caso, que estamos trabajando sobre Linux y también sobre Ruby
podemos instalarlo de la siguiente manera:

\begin{verbatim}
gem install heroku foreman
\end{verbatim}

Una vez instalado, para tener acceso a la línea de comando de heroku desde tu shell habrá que hacer un \"login\":

\begin{verbatim}
$ heroku login
Enter your Heroku credentials.
Email: adam@example.com
Password:
\end{verbatim}

Si no tenemos ninguna clave de ssh, nos preguntará que si deseamos crear una:
\begin{verbatim}
Could not find an existing public key.
Would you like to generate one? [Yn]
Generating new SSH public key.
Uploading ssh public key /Users/adam/.ssh/id_rsa.pub
\end{verbatim}

Ahora ya estamos preparados para crear nuestra primera aplicación en Heroku.
Para ello nos tendremos que colocar en el directorio donde tengamos la aplicación: 

\begin{verbatim}
$ cd ~/myapp
\end{verbatim}

y una vez en dicho directorio teclear:

\begin{verbatim}
$ heroku create
Creating stark-fog-398... done, stack is cedar
http://stark-fog-398.herokuapp.com/ | git@heroku.com:stark-fog-398.git
Git remote heroku added
\end{verbatim}

\section{Despliegue con Heroku} %{\tt system}}
Heroku nos permite desplegar aplicaciones desarrolladas en lenguajes tales como Java, Phyton, Ruby, etc. En nuestro caso, haremos el despligue
en Heroku para una aplicación desarrollada en Rails.
\newline
Una vez hecho el login en Heroku (\verb|heroku login|), nos colocaremos en el directorio donde esté nuestra aplicación y modificaremos nuestro
\verb|Gemfile| cambiando esta línea:

\begin{verbatim}
gem 'sqlite3'
\end{verbatim}

por esta otra:

\begin{verbatim}
gem 'pg'
\end{verbatim}

Ahora reinstalamos nuestras dependencias tecleando:

\begin{verbatim}
 $ bundle install
\end{verbatim}
Esta modificación se hace debido a que Rails suele trabajar con \textbf{Sqlite3} y Heroku lo hace con \textbf{PostgreSQL}.

*Nota: Para que no falle la instalación de \verb|pg| hay que tener instalada la librería libpq-dev

\subsection{}

Continuaremos almacenando nuestra aplicación en Git, para ello teclearemos:
\begin{verbatim}
$ git init
$ git add .
$ git commit -m "Iniciando..."
\end{verbatim}

Ahora creamos la aplicación en la pila Cedar de Heroku:

\begin{verbatim}
$ heroku create
Creating severe-mountain-793... done, stack is cedar
http://severe-mountain-793.herokuapp.com/ | git@heroku.com:severe-mountain-793.git
Git remote heroku added
\end{verbatim}

Y realizamos el despliegue de nuestro código tecleando:

\begin{verbatim}
$ git push heroku master
\end{verbatim}

y nos aparecerá algo similar a esto:

\begin{verbatim}
Counting objects: 67, done.
Delta compression using up to 4 threads.
Compressing objects: 100% (52/52), done.
Writing objects: 100% (67/67), 86.33 KiB, done.
Total 67 (delta 5), reused 0 (delta 0)

-----> Heroku receiving push
-----> Rails app detected
-----> Installing dependencies using Bundler version 1.1
       Checking for unresolved dependencies.
       Unresolved dependencies detected.
       Running: bundle install --without development:test --path vendor/bundle --deployment
       Fetching source index for http://rubygems.org/
       Installing rake (0.8.7)
       ...
       Installing rails (3.0.5)
       Your bundle is complete! It was installed into ./vendor/bundle
-----> Rails plugin injection
       Injecting rails_log_stdout
       Injecting rails3_serve_static_assets
-----> Discovering process types
       Procfile declares types -> (none)
       Default types for Rails -> console, rake, web, worker
-----> Compiled slug size is 8.3MB
-----> Launching... done, v5
       http://severe-mountain-793.herokuapp.com deployed to Heroku

To git@heroku.com:severe-mountain-793.git
 * [new branch]      master -> master
\end{verbatim}

Antes de comprobar nuestra aplicación en la web, debemos comprobar los procesos de la aplicación de la siguiente manera:

\begin{verbatim}
 $ heroku ps
=== web: `bundle exec rails server -p $PORT`
web.1: up for 5s
\end{verbatim}

Ahora es el momento de realizar la migración de la base de datos, para que empieze a funcionar con PostgreSQL:

\begin{verbatim}
$ heroku run rake db:migrate
\end{verbatim}

Por defecto la aplicación correrá con el \verb|rails server| que usa Webrick. Esto es conveniente para el testeo, pero para producción
necesitaremos el uso de un servidor más robusto. Para ello incluimos en nuestro \verb|Gemfile|
\begin{verbatim}
 gem 'thin'
\end{verbatim}
ya que en la propia página de Heroku está recomendado. Completamos la instalación local de la gema tecleando en consola \verb|bundle install|

Ahora para cambiar el comando que usamos para lanzar los procesos web creamos un fichero llamado \"Procfile\" e insertamos esto:
\begin{verbatim}
 web: bundle exec rails server thin -p $PORT -e $RACK_ENV
\end{verbatim}
configuramos la variable de entorno:
\begin{verbatim}
 $ echo "RACK_ENV=development" >>.env
\end{verbatim}

Ahora probaremos el \verb|Procfile| localmente usando \textbf{``Foreman''}. Para ello tecleamos:
\begin{verbatim}
 $ foreman start
11:35:11 web.1     | started with pid 3007
11:35:14 web.1     | => Booting thin
11:35:14 web.1     | => Rails 3.0.4 application starting in development on http://0.0.0.0:5000
11:35:14 web.1     | => Call with -d to detach
11:35:14 web.1     | => Ctrl-C to shutdown server
11:35:15 web.1     | >> Thin web server (v1.2.8 codename Black Keys)
11:35:15 web.1     | >> Maximum connections set to 1024
11:35:15 web.1     | >> Listening on 0.0.0.0:5000, CTRL+C to stop
\end{verbatim}
Si nos sale algo como lo mostrado habrá ido todo correcto. Presionamos Ctrl + C para salir y desplegamos los cambios en heroku:

\begin{verbatim}
 $ git add .
$ git commit -m "use thin via procfile"
$ git push heroku
\end{verbatim}

Al comprobar el log, veremos que ahora usamos \textbf{``Thin''}

\begin{verbatim}
 $ heroku logs
2012-07-11T09:39:36+00:00 heroku[web.1]: State changed from down to starting
2012-07-11T09:39:38+00:00 heroku[web.1]: Starting process with command `bundle exec rails server thin -p 22617 -e production`
2012-07-11T09:39:44+00:00 app[web.1]: => Booting Thin
2012-07-11T09:39:44+00:00 app[web.1]: => Rails 3.2.6 application starting in production on http://0.0.0.0:22617
2012-07-11T09:39:44+00:00 app[web.1]: => Call with -d to detach
2012-07-11T09:39:44+00:00 app[web.1]: => Ctrl-C to shutdown server
2012-07-11T09:39:44+00:00 app[web.1]: Connecting to database specified by DATABASE_URL
2012-07-11T09:39:44+00:00 app[web.1]: >> Thin web server (v1.4.1 codename Chromeo)
2012-07-11T09:39:44+00:00 app[web.1]: >> Maximum connections set to 1024
2012-07-11T09:39:44+00:00 app[web.1]: >> Listening on 0.0.0.0:22617, CTRL+C to stop
2012-07-11T09:39:45+00:00 heroku[web.1]: State changed from starting to up

\end{verbatim}



